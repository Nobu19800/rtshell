\documentclass[a4paper,10pt]{article}

\usepackage{CJKutf8}
% \usepackage[UKenglish]{babel}
\usepackage[overlap, CJK]{ruby}
\usepackage{CJKulem}

\renewcommand{\rubysep}{-0.2ex}
\newenvironment{Japanese}{%
  \CJKfamily{min}%
  \CJKtilde
  \CJKnospace}{}

\usepackage{pdfsync}
\usepackage{a4wide}
\usepackage{url}
\usepackage{graphicx}
\usepackage{booktabs}
\usepackage{tabularx}
\usepackage{listings}

\lstloadlanguages{Bash}
\lstset{basicstyle=\small\ttfamily, escapeinside={(*@}{@*)}, showstringspaces=false, numbers=none,
breaklines=true, breakatwhitespace=true, tabsize=4, language=Bash, numberfirstline=true}

\begin{document}
% \begin{CJK}{UTF8}{}
% \begin{Japanese}

\begin{table}
  \centering
  \begin{tabularx}{\columnwidth}{ll}
    \toprule
    Command & Purpose \\
    \midrule
    rtact $\langle$component$\rangle$ & Activate a component. \\
    rtcat & Display component meta-data. \\
    rtcon $\langle$component 1$\rangle$:$\langle$port 1$\rangle$ & Connect two ports. \\
    \qquad$\langle$component 2$\rangle$:$\langle$port 2$\rangle$ & \\
    rtconf $\langle$component$\rangle$ & Display, select and edit configuration sets and parameters. \\
    rtcwd [path] & Change the current working directory in the RTC Tree. \\
    rtdeact $\langle$component$\rangle$ & Deactivate a component. \\
    rtdel $\langle$path$\rangle$ & Delete an object from a naming context. \\
    rtdis $\langle$component 1$\rangle$[:port 1] & Disconnect two ports, or all connections from a port or \\
    \qquad[component 2[:port 2]] & component. \\
    rtfind $\langle$path$\rangle$ $\langle$options$\rangle$ & Search for components and other objects in the RTC Tree. \\
    rtinject $\langle$component$\rangle$:$\langle$port$\rangle$ $\langle$data$\rangle$ & Send data to an InPort on a component. \\
    rtls [path] & Display the contents of a directory in the RTC Tree. \\
    rtmgr $\langle$manager$\rangle$ & Control a manager. \\
    rtprint $\langle$component$\rangle$:$\langle$port$\rangle$ & Print the data being sent by an OutPort in the console. \\
    rtpwd & Print the current working directory in the RTC Tree. \\
    rtreset $\langle$component$\rangle$ & Reset a component. \\
    \bottomrule
  \end{tabularx}
\end{table}

\begin{table}
  \centering
  \begin{tabularx}{\columnwidth}{ll}
    \toprule
    Command & Purpose \\
    \midrule
    rtresurrect $\langle$xml|yaml file$\rangle$ & Restore a complete RT System. \\
    rtstart $\langle$xml|yaml file$\rangle$ & Start an RT System. \\
    rtstop $\langle$xml|yaml file$\rangle$ & Stop an RT System. \\
    rtcryo $\langle$xml|yaml file$\rangle$ & Save an RT System to a file. \\
    rtteardown $\langle$xml|yaml file$\rangle$ & Delete all connections in an RT System. \\
    \bottomrule
  \end{tabularx}
\end{table}

\newpage

\section*{Try out these commands}

\begin{lstlisting}[frame=tb]
1.rtls
2.rtcwd localhost/me.host_cxt/
3.rtls
4.rtfind . --type=m
5.rtfind . --type=c
6.rtls -l
7.rtmgr manager.mgr load /usr/local/share/OpenRTM-aist/examples/rtcs/Sensor.so SensorInit
8.rtcat manager.mgr
9.rtmgr manager.mgr create Sensor
10.rtls / -R
11.rtcon Motor0.rtc:out ../Sensor0.rtc:in
12.rtcon ../Sensor0.rtc:out Controller0.rtc:in
13.rtcon Motor0.rtc:in Controller0.rtc:out
14.rtls -l
15.rtcon ConsoleIn0.rtc:out ConsoleOut0.rtc:in
16.for c in `rtfind . --type=c`; do rtact ${c}; done
17.rtls -l
18.rtinject ConsoleOut0.rtc:in 'RTC.TimedLong({time}, 42)'
19.rtprint ConsoleIn0.rtc:out
20.rtcat ConsoleOut0.rtc
21.rtcat ConsoleOut0.rtc -l
22.rtcat ConsoleOut0.rtc --ll
23.rtconf ConfigSample0.rtc
24.rtconf ConfigSample0.rtc -l
25.rtconf ConfigSample0.rtc set default int_param0 42
26.rtconf ConfigSample0.rtc -l
27.watch -n 1 rtls -l
\end{lstlisting}

\vspace{1cm}

\begin{lstlisting}[frame=tb]
1.rtls -l
2.rtresurrect --dry-run rtsystem.xml
3.rtresurrect rtsystem.xml
4.rtls -l
5.rtstart --dry-run rtsystem.xml
6.rtstart rtsystem.xml
7.rtls -l
8.rtls -l ../
9.rtstop rtsystem.xml
10.rtls -l
11.rtcryo localhost -o sys.xml
12.rtteardown sys.xml
13.rtls -l
\end{lstlisting}

% \end{Japanese}
% \end{CJK}
\end{document}
